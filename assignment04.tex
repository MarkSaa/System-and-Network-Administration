\input macros
\rightskip=0pt plus 1fill
\input cstuff
\headline{{\bf CECS 476\hfill Assignment 4\hfill Spring 2019}}
\footline{Nathan Pickrell \hfill 5 February 2019 (Week 3 Lecture 1)\hfill 
Due: 7 February 2019 (Week 3 Lab 2)}
\parindent 0pt

On all assignments, be sure to indicate the name of the
machine you are assigned to administer.

This first question of this assignment emphasizes shell programming,
the remaining questions have you examine the file system.

Make sure the user {\ltt{}bob} exists on your 
machine.

Simple shell script.

As bob, write a shell script that examines all the processes on the
system and ``reports" bash shell processes
into a file called {\ltt{}bashlog}.
First, this script should append to the {\ltt{}bashlog} the {\ltt{}date}.
Then, it should do a {\ltt{}ps aux}, and all lines (use {\ltt{}grep})
containing the letters bash but not containing the letters cpeek
should be appended to the {\ltt{}bashlog}.
Your script file should be called {\ltt{}cpeek}

Test your shell script by
running your command several times with various users logged 
in (for example root and your csa account)
(i.e., just make sure it works).

1) Report: the exact contents of {\ltt{}cpeek}.
(Do NOT report the contents of your bashlog.)

#!/bin/bash
date >> bashlog.txt
ps aux | grep bash | grep -v "cpeek" >> bashlog.txt

More advanced shell script.

As bob, write a shell script called {\ltt{}clook}.
This shell script takes an argument that should be the name of the directory.
If it is given the name of something that isn't a directory (or doesn't
exist) it should print ``Sorry".
If it is given the name of a directory it should print the names of
any items in that directory that are files and are executable. 

Test your shell script by running your command on {\ltt{}/etc/printcap}
(it's a file and should print ``Sorry") and on {\ltt{}~djv} (do an {\ltt{}ls}
to see confirm the names of the executables.

2) Report: the exact contents of {\ltt{}clook}.

#!/bin/bash
test -d $1
	if [ $? == 0 ]; then
		for file in $(ls -l)
			do
				if [ -x "$file" ] && [ -f "file" ]; then
				echo $file
			fi
		done
	else
	echo "Sorry"
fi

On the {\ltt{}cheetah} examine and report the following:

3) What three hard drives are attached to the file tree and where (mount/df)?
With the df command, Linux lists one hard drive as a virtual device called 
{\ltt{}/dev/root}. You can determine which hard drive this is by doing a 
{\ltt{}mount} command and looking for the entry for ``/".

ls -l /dev/root


lrwxrwxrwx 1 root root 4 Feb  5 19:10 /dev/root -> sda2

df

/dev/sdb1                480589544  47465688 408688196  11% /u3
/dev/sdc1                480589544 293220700 162933184  65% /u4

4) On the root ({\ltt{}/}) file system, how much disk space is available (df)?

df

lesystem                           1K-blocks      Used Available Use% Mounted
on
/dev/root                            865292600   6229316 815108972   1% /
devtmpfs                                995744         0    995744   0% /dev
tmpfs                                   997968       888    997080   1% /run
tmpfs                                   997968         0    997968   0%
/dev/shm
cgroup_root                             997968         0    997968   0%
/sys/fs/cgroup
/dev/sdb1                            480589544  47465696 408688188  11% /u3
/dev/sdc1                            480589544 293220700 162933184  65% /u4
cgmfs                                      100         0       100   0%
/run/cgmanager/fs
jaguar:/sdb/slack14.2-64/slackware64 153836544  18246400 127775744  13%
/mnt/cdrom

5) On the root ({\ltt{}/}) file system, what is the file system type
and is the file system read only or read/write (mount)?

The filesystem type is ext4 and is (rw) read/write

mount

/dev/sda2 on / type ext4 (rw)
proc on /proc type proc (rw)
sysfs on /sys type sysfs (rw)
tmpfs on /dev/shm type tmpfs (rw)
/dev/sdb1 on /u3 type ext4 (rw)
/dev/sdc1 on /u4 type ext4 (rw)
nfsd on /proc/fs/nfs type nfsd (rw)
nfsd on /proc/fs/nfsd type nfsd (rw)
jaguar:/sdb/slack14.2-64/slackware64 on /mnt/cdrom type nfs
(rw,addr=134.139.248.18)

6) In your home directory, how much space have you used (not much yet) (du)?

du

8	./.ssh
72	.


7) Your home directory is on one of the hard drive partitions.
Report the line in the {\ltt{}fstab} that causes that partition
to be mounted.

cat /etc/fstab

/dev/sda1        swap             swap        defaults         0   0

On the system you administer, using the system administrator account;
examine the superblock of the linux partition on your hard drive
({\ltt{}/dev/sda2}) using the {\ltt{}dumpe2fs} command.

8) Report: The file system state, the block size and the number of groups
in that file system.

dumpe2fs /dev/

Filesystem state:         not clean
Block size:               4096
Group 476:

9) {\bf fdisk -l}:
For the disk you administer describe the partitions that are on the disk?
For each partition {\it report}, what cylinder it starts at,
what cylinder it ends at,
and what kind of a partition it is (Linux, DOS, swap).
Be very careful with this one, 
DO NOT MODIFY or WRITE anything with fdisk; 
you could erase your hard drive.

fdisk -l

Device     Boot    Start       End   Sectors  Size Id Type
/dev/sda1           2048  31258180  31256133 14.9G 82 Linux swap
/dev/sda2       31258181 156301487 125043307 59.6G 83 Linux

Submission: You will submit hw04.txt with ``\tilde grader/submit 04". 
It must contain the contents of cpeek and clook as well as answers to 
the other questions.

\bye
