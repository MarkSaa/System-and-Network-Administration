\input macros
\rightskip=0pt plus 1fill
\input cstuff
\headline{{\bf CECS 476\hfill Assignment 11 \hfill Spring 2019}}
\footline{Nathan Pickrell \hfill 28 February 2019 (Week 6 Lecture 2)\hfill 
Due: 5 March 2019 (Week 7 Lab 1)}
\parindent 0pt
% INSTRUCTOR NOTE: run chmod a-x /etc/rc.d/rc.inet1
1) This project is about configuring a machine to be on the network.

Before starting this project, ask the instructor to disable networking
on your machine. He will have to reboot to do this.

Get you machine up and onto the network.
Do this by typing commands in one at a time, not by running a script.
You will not enable all of networking, just enough to be
able to ping {\ltt{}127.0.0.1}, {\ltt{}134.139.248.65},
{\ltt{}134.139.248.33} and {\ltt{}134.139.248.17}.
To do this you need to get the interface and route tables up.

The hostname, internet address, gateway and netmask of your machine are
on the label on the front.
The cable number for your machine is its address with the host
part set to all zeros.
The broadcast address for your machine is its address with the host
part set to all ones.
You should test to ensure your machine is connected to the network.

Report: the exact sequence of commands you used to connect your machine
to the network.

--root--

ifconfig eth0 134.139.248.70 broadcast 134.139.248.95 netmask 255.255.255.224

route add -net 134.139.248.64 netmask 255.255.255.224 gw 134.139.248.65 metric
1

ifconfig eth0 134.139.248.70 netmask 255.255.255.224
route add default gw 134.139.248.65

--type in command and restart--
chmod a+x /etc/rc.d/rc.inet*


Clean up: to re-enable default networking you must change the mode
of the network initialization files used by {\ltt{}init}.
Type the following command:
\hfill\break
{\ltt{}chmod a+x /etc/rc.d/rc.inet*}
\hfill\break
Now when you reboot {\ltt{}init} will bring the network up automatically.

2) Resolver/DNS. The commands {\ltt{}nslookup www-main} and 
{\ltt{}nslookup columbo} fail because the full
names of the machines are {\ltt{}www-main.ics.uci.edu} and 
{\ltt{}columbo.ics.uci.edu}.
Fix your machine so that {\ltt{}nslookup www-main} works (as well as using
the ``first-name" of all other machines ending in {\ltt{}ics.uci.edu}).
Sanity check, make sure {\ltt{}nslookup} still works for {\ltt{}panther}
and {\ltt{}cheetah}

vim /etc/resolv.conf

--add to the end of search--

ics.uci.edu


3) Using the {\ltt{}inetd.conf}, disable telnets into your machine.

Report: What did you do to the file.

Cleanup: reenable telnets.

vim /etc/inetd.conf

--comment out the telnet command--
--save and exit--
--go back in and undo--

4) Using the tcp wrappers, deny telnets from {\ltt{}panther}. (Be sure to test
that they are still allowed from other machines.)
The manual entry for this question is {\ltt{}hosts_access} from section 5.

Report: Which file did you use and what did you put into it.

Cleanup: reenable telnets.

vim /etc/hosts.deny

in.telnetd: panther.net.cecs.csulb.edu

--test and undo--


5) Report: List the names of the remote programs available on {\ltt{}jaguar}.
(List each name just once).

rpcinfo -p jaguar

portmapper
status
nlockmgr
ypbind
rquotad
nfs
mountd

6) Allow {\ltt{}csa476xx} (your account) on {\ltt{}panther} to rlogin as 
{\ltt{}bob} on the machine you administer without a password.

Report: Which file did you use, on which machine, and what did you put into it.

vim ~bob/.rhosts

panther.net.cecs.csulb.edu csa476t1

7) Allow any user from {\ltt{}puma} to rlogin to the machine you administer 
without a password.
That is, if they are {\ltt{}joe} on {\ltt{}puma}, they are allowed to
rlogin as {\ltt{}joe} on your machine without a password.
Of course {\ltt{}joe} must be  a user that exists.
Check this out by using your {\ltt{}csa476xx} account (because it is the only
account you have access to that exists on both machines).

Report: Which file did you use, on which machine, and what did you put into it.

vim /etc/hosts.equiv

puma.net.cecs.csulb.edu

--test: shh to puma--
--test with: rlogin lab70 -l csa476t1--


8) Allow {\ltt{}csa476xx} (your account) on {\ltt{}panther} to ssh to
{\ltt{}bob} on the machine you administer without a password.

Report: the steps you took to allow this.

--ssh to panther--

ssh-keygen -t -rsa

~/.ssh$ scp id_rsa.pub bob@lab70:newkey

--ssh bob@lab70--

ssh-keygen -t -rsa

cat newkey >> ~/.ssh/authorized_keys



\bye
