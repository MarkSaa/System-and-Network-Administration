\input macros
\rightskip=0pt plus 1fill
\input cstuff
\headline{{\bf CECS 476\hfill Assignment 12 \hfill Spring 2019}}
\footline{Nathan Pickrell \hfill 5 March 2019 (Week 7 Lecture 1)\hfill 
Due: 7 March 2019 (Week 7 Lab 2)}
\parindent 0pt

Since NIS is running, this project mostly involves only
information gathering and configuration examination:

1) What is the NIS domain name of your machine.
Report the command you ran to find this out and the domain name.

--in root--

cat /etc/defaultdomain

cecsnet


2) What NIS server is your machine using.
Report the command you ran to find this out and the server name.

ypwhich

cheetah.cecs.csulb.edu


3) How many password lines does NIS deliver on your machine.
Report: the exact command you used and the number of lines.

ypcat passwd | wc -l

51

4) Examine the NIS password information arriving at your machine.
For your 476 account {\it report} the exact line of 
information that arrives.
You are NOT allowed to use the same yp command you used in the previous
question, use another command to match the line in the password file.
Report the exact command you used to get that information.

ypmatch csa476t1 -k passwd

csa476t1
csa476t1:$6$C/x3S/t0$7HPe7IgYLBNyMXf40F2x1eBQrCFWKYudqFezIMHDpTORsG5ZaKRoqLR3kcd95fQwlqS5IegFsNxHCz5xOqx0N0:20062:1001:student:/net/cheetah/u3/c/csa476/t1:/bin/bash


5) Examine the {\ltt{}/var/yp} directory tree on your machine.
Report: what in this subtree is dependent upon your domain name.

vim /var/yp/binding

cecsnet.1
cecsnet.2

6) Examine your {\ltt{}yp.conf} file, what is there.

cat /etc/yp.conf

--commented lines--
ypserver 134.139.248.17


7) Examine the start-up code for the yp client program.
What is the exact (full) pathname of the program that is run
(that means the name starting with the {\ltt{}/}).
Under what condition is this program started?

vim /etec/rc.d/rc.yp

if [ $YP_CLIENT_ENABLE -new 0]; then
if [ -d /var/yp ]; then
echo "Starting NIS services: /usr/sbin/ypbind"
/usr/sbin/ypbind
fi 
fi


8) Examine your {\ltt{}nsswitch.conf}.
Is it set up to use yp for passwords?
How do you know?

vim /etc/nsswitch.conf

passwd:	files nis

Yes, it is set up for NIS

9)  On cheetah, examine the {\ltt{}/etc/netgroup} file.
Report: How many machines are in the {\ltt{}cslabd} netgroup.

--in cheetah--

vim /etc/netgroup

cslabd (panther, -,) (puma,-,) (lynx,-)


10) On the machine you administer, the groups that {\ltt{}cheetah} is
delivering using NIS are being used. 
Do the following.
Turn off the use of NIS groups on your machine.
(A HUP is not necessary in this case.)
An {\ltt{}ls -l ~sue} should give the number 1202
instead of a name for the ``group" of sue's files.
When this works, turn back on the use of NIS groups.
Again use the {\ltt{}ls -l ~sue}, this time to make sure
that a name is given for the group of sue's files.
Report: What did you do to turn on/off the use of NIS groups?


vim /etc/nsswitch.conf

--comment out nis in group:--

ls -l ~sue
total 4
drwxr-xr-x 2 sue 1202 4096 Feb  4 16:44 skel/

--comment nis back into group:--

ls -l ~sue
total 4
drwxr-xr-x 2 sue cecsg 4096 Feb  4 16:44 skel/

\bye
