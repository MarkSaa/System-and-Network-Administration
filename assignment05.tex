\input macros
\rightskip=0pt plus 1fill
\input cstuff
\headline{{\bf CECS 476\hfill Assignment 5 \hfill Spring 2019}}
\footline{Nathan Pickrell \hfill 7 February 2019 (Week 3 Lecture 2)\hfill 
Due: 12 February 2019 (Week 4 Lab 1)}
\parindent 0pt

Purpose: This assignment covers disk and file system administration.

Submission: You will submit hw05.txt using ``\tilde grader/submit 05" as before.
Your instructor will also test your configurations while grading.

1) On the machine you administer there is a second drive.
It should be {\ltt{}/dev/sdb}, if you don't locate such a disk
let the instructor know.

Run {\ltt{}fdisk} on {\ltt{}sdb}.
Remove all existing partitions and create two partitions.
The first partition should use about 10 percent of the
disk, and the second partition should contain the remainder.
Change the type of the first partition to linux swap.
(You can have {\ltt{}fdisk} give you a list of partition codes that it uses.)

While running {\ltt{}fdisk} make sure the second parition
you created is marked as ``linux native" or ``linux", this is almost 
always the default when you make a partition, but it is always worth checking
when using a version of {\ltt{}fdisk} you haven't used before.
Save your partition table.

Run {\ltt{}mke2fs} to build 
an ext2 file system on the second disk partition (the large
one) and attach that file system as {\ltt{}/newdisk}.
Be careful not to modify (or damage) your main drive
which is {\ltt{}sda} or you will destroy your system.

Now mount the file system you just created by hand
(this is the safest way to make sure it is set up right).

Submit: 
1a) a brief description what you did with {\ltt{}fdisk} and
a listing of your partition table.

fdisk /dev/sdb
command: n -- create partition
command: p -- for primary partition
command: 1 -- for partition number

repeated steps to create another partition

command: t -- to change the partition type
command: 1 -- to go to partition 1
command: 82 -- for hex code Linux Swap
command: w -- to write table to disk


1b) the exact form of the {\ltt{}mount} command you used and
the disk free {\ltt{}df} report on the new file system.

mke2fs sdb2
mkdir /newdisk
mount -w /dev/sdb2 /newdisk
df -h /newdisk

Filesystem	Size	Used	Avail	Use%	Mounted on
/dev/sdb2	67G	52M	63G	1%	/newdisk

2) Modify your system so that the mount will 
occur automatically at boot.
Test this by rebooting the machine.

Submit: the line you added to your {\ltt{}fstab}.

vim /etc/fstab

/dev/sdb2	/newdisk	ext2	defaults	1	1

--restart--

3) You will need a floppy disk or flashdrive for questions 3 and 4. Set-up 
an ext2 file system on this device (similar process to setting up second disk);
Make an entry in {\ltt{}fstab} that allows any user to insert and 
mount an ext2 floppy/flashdrive using {\ltt{}/mnt/extdisk} as the mount point.
Change the permissions on {\ltt{}/mnt/extdisk} to 777 
({\ltt{}chmod 777 /mnt/extdisk}).
Login as bob (not root), plug in your ext2 device, 
mount it, copy a file to the device, unmount and remove the device.

Submit: the line you entered into {\ltt{}fstab} and
the exact mount command you used.

mke2fs /dev/sdc1
mkdir /mnt/extdisk
mount -w /dev/sdc1 /mnt/extdisk

vim /etc/fstab

/dev/sdc1	/mnt/extdisk	ext2	user,noauto	0	0

chmod 777 /mnt/extdisk

chown -R bob:1201 /mnt/extdisk

--login as bob--
mount /mnt/extdisk

4) As the system administrator: run an {\ltt{}fsck} on your device. 
{\it Remember:} Unix/Linux usually wants the file system being checked to 
be unmounted, so it is not changing while the check is performed. 
If the file system is clean you may need to force a check ({\ltt{}-f} option)
 to see one actually occur.

Submit: the exact command you used and summarize what the command output.

umount /dev/sdc1
fsck /dev/sdc1

chmod 755 /mnt

Clean up: Change the permissions on {\ltt{}/mnt} back to 755.
Unmount the device before removing and returning it.
Leave the hard drive mounted, leave the fstab for the instructor to examine.

Note: the first smaller partition you created on your harddrive
in this assignment will be used in the next assigment.
\bye
