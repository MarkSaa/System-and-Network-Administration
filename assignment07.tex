\input macros
\rightskip=0pt plus 1fill
\input cstuff
\headline{{\bf CECS 476\hfill Assignment 7 \hfill Spring 2019}}
\footline{Nathan Pickrell \hfill 14 February 2019 (Week 4 Lecture 2)\hfill 
Due: 21 February 2019 (Week 5 Lab 1)}
%lecture 7
\parindent 0pt

Purpose: This part of the assignment covers init.

Submission: You will submit with ``\tilde grader/submit 07".
You will also provide a ``demonstration".

You want to attach a terminal to serial port 0 (Windows calls this COM1).
(1) Report: What line would you need to change in the inittab.
(Do not make this change, just report the line that would need
to be changed.)

The following must be done from CONSOLE (at the keyboard of your machine):

Modify your {\ltt{}inittab} so that control-alt-delete does not
reboot the machine.
Cause {\ltt{}init} to re-read {\ltt{}inittab}.
Check to make sure that your have successfully disabled control-alt-delete.

Good practice: never delete or modify a default configuration line, 
comment the old one out (so you can go back to it) and add the new one.

{\it Demonstration}: 
call the instructor over and let him type control-alt-delete
(your machine should not reboot).

Cleanup: restore {\ltt{}inittab} to it's original condition. 


In root

command:
vim /etc/inittab

s1:12345:respawn:/sbin/agetty -L ttyS0 9600 vt100


(2) Report: The exact text of the line that disables control-alt-delete.
The command you ran to make that line take effect

init q


ca::ctrlaltdel:/sbin/shutodwn -t5 -r now


On shutdown, how many seconds does {\ltt{}init} wait before 
forcibly terminating processes? (Hint: {\ltt{}man init})
\break
3) Report: the number of seconds.

It waits 5 seconds before forcibly terminating these processes via the SIGKILL
signal

This part of the assignment cron and at.

Cron: as bob; set up you crontab so that you run an {\ltt{}ls} command
every minute (redirect the output of the {\ltt{}ls} to a file). 

4) Report: your crontab entry. Just the one line is ok.

crontab -e

* * * * * ls >> ~/lslist.txt

Clean up: remove your crontab entry.

crontab -d 

--removed--

At: as bob; Set up an at entry, that in 5 minutes from now will
run the {\ltt{}ls} command.
(Again, redirect the output of the {\ltt{}ls} to a file.) 

5) Report: the exact command you used to do this.

at 18:16
at> ls >> ~/lslist2.txt 

\bye
