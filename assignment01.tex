\input macros
\rightskip=0pt plus 1fill
\input cstuff
\headline{{\bf CECS 476\hfill Assignment 1 \hfill Spring 2019}}
\footline{Nathan Pickrell \hfill 24 January 2019 (Week 1 Lecture 2)\hfill 
Due: 29 January 2019 (Week 2 Lab 1)}
\parindent 0pt

Purpose:
To familiarize you with the systems you will be using.

Submission: You will submit a document (text file) called {\ltt{}hw01.txt} 
via the command ``\tilde grader/submit 01".
As a reminder, this document and all submissions in this course will 
have a header containing the following information:

Your name, your login name, and the name of the machine on which you
did the assignment (see label on the front, or use hostname command).

All the questions will also serve to drill you on the organization
of the Linux file system.
The first three questions are also a drill on using the online manual.

How many users are listed in the password file of the server cheetah?
(Fortunately there is one user per line.)
Hint: use {\ltt{}man -k} to find a command that prints the number
of lines in a file.

1) Report: the number of entries in the password file and
the command you used to find out.

man -k number of lines
nl /etc/passwd

File system organization:

Examine: the configuration file that controls the {\ltt{}inetd} system service
on your Linux system.

2) Report: 
What is the exact name of this file,
in what directory did you find this file,
and what commands did you use to do this.

File name: -rw-r--r-- 1 root root 4583 Jan 12 11.57 /etc/inetd.conf

locate inetd
ls -l /etc/inetd.conf

Locate: the program that will tell the disk space usage
on your Linux system.

3) Report: What is the name of that program,
in what directory (where) is the program found
and what commands did you use to do this.

man -k disk space usage
which df
df

The next 5 questions are designed to make you use various
commands with specific options.
The answers themselves are of minor interest, the drill here is
on use of various Unix commands.
These commands will be frequently used by you in later assignments.

Report the answer to each of the following questions and the
commands you used to find that answer.

4) Your {\ltt{}/etc/X11} directory should contain templates for
the configuration file for the {\ltt{}X} window system,
what are the names of these files/this file?
(Hint: the file names start with xorg.)

cd /etc/X11
ls xorg*

Name of files:
xorg.conf-vesa
xorg.conf.d:

5) In the {\ltt{}/usr/lib64} directory, there are several entries that support
the {\it Xfontcache} (the library name includes those letters). What are the
full names of those entries?
(Hint, use ls with wild cards).

cd /usr/lib64
ls *Xfontcache*

Full name of entries: 
libXfontcache.la*
libXfontcache.so@
linXfontcache.so.1@
libXfontcache.so.1.0.0*

6) Examine the {\ltt{}/proc} file system.
What is the cpu MHz of your machine.
Hint: this is information about the cpu and can be obtained using the
cat command on the correct pseudo-file.

cd /proc
cat cpuinfo

cpu MHz of machine: 1200.00

Examine the log files on the server{\ltt{}jaguar} (Hint: You will
need to ssh or telnet to jaguar):
\hfill\break
7) Give the names of the last 3 log files that have changed and 
the dates at which they changed.
\hfill\break

cd /usr/log/
ls -ltr

log files:
-rw-r--r-- 1 root root 350815 Jan 24 18:38 messages
-rw -rw-r-- 1 root utmp 79488 Jan 24 18:38 wtmp
-rw-r--r-- 1 root root 5865696 Jan 24 18.38 lastlog


8) Give the last 3 lines found in the {\ltt{}messages} log file.

ssh jaguar
cd /var/log/
ls -ltr

Jjan 24 18:40:47 jaguar sshd[3743]: Accepted password for csa76u1...
Jan 24 18:41:02 jaguar sshd[3762]: Accepted password for csa76s1
Jan 24 18:41:34 jaugar sshd[3758]: Accepted password for csa76g1

If you have NOT run the register program from hw00, please do 9 to
be added to my gradesheet.

9) Log into {\ltt{}cheetah} and run the {\ltt{}register} program,
follow the intructions that program gives you.
(No submission on this one.)

\bye
