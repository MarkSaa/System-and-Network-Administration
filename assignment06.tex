\input macros
\rightskip=0pt plus 1fill
\input cstuff
\headline{{\bf CECS 476\hfill Assignment 6 \hfill Spring 2019}}
\footline{Nathan Pickrell \hfill 12 February 2019 (Week 4 Lecture 1)\hfill 
Due: 14 February 2019 (Week 4 Lab 2)}
\parindent 0pt

This part of the assignment is designed to familiarize you with the
commands involving Unix processes.

Submission: You will submit hw06.txt with ``\tilde grader/submit 06".
Your instructor will also test your swap configurations while grading.

For Cheetah: 1) how long has it been since it has
been rebooted, 

uptime

17:07:47 up  1:58

2) how many users are logged in, 

24 users

3) how busy is the machine
(short, medium and long term load).

load average: 0.00, 0.00, 0.00

For Cheetah: 4) What percentage of the cpu is the operating system
spending in each of user mode, system mode and idle?

top

us,  0.7% 
sy,  0.5%
id,  98.7%


5) How much memory is in use and
how much is idle?

Used: 198448 used
Idle: 1517648 free

6) What 3 processes have the largest resident set size?

PID USER      PR  NI    VIRT    RES    SHR S  %CPU %MEM     TIME+ COMMAND    
 6682 csa476d2  20   0   14368   2740   2136 S   1.0  0.1   0:01.80 top        
 6943 csa476t1  20   0   14336   2760   2312 R   0.7  0.1   0:00.66 top        
 7062 csa476u1  20   0   14340   2768   2136 S   0.7  0.1   0:00.28 top

For the machine you administer answer the following:
7) How many processes called {\ltt{}agetty} are running?

ps auxww | grep agetty

root      1302  0.0  0.0   6948  1480 tty1     Ss+  15:09   0:00 /sbin/agetty
38400 tty1 linux
root      1303  0.0  0.0   6948  1636 tty2     Ss+  15:09   0:00 /sbin/agetty
38400 tty2 linux
root      1304  0.0  0.0   6948  1556 tty3     Ss+  15:09   0:00 /sbin/agetty
38400 tty3 linux
root      1305  0.0  0.0   6948  1644 tty4     Ss+  15:09   0:00 /sbin/agetty
38400 tty4 linux
root      1306  0.0  0.0   6948  1728 tty5     Ss+  15:09   0:00 /sbin/agetty
38400 tty5 linux
root      1307  0.0  0.0   6948  1708 tty6     Ss+  15:09   0:00 /sbin/agetty
38400 tty6 linux
csa476t1  7358  0.0  0.1  11672  2100 pts/6    S+   17:58   0:00 grep agetty

8) There are several commands running that include the letters {\ltt{}rpc},
list all those commands.

ps auxww | grep rpc

root        28  0.0  0.0      0     0 ?        S<   15:08   0:00 [rpciod]
bin        966  0.0  0.1  13188  2292 ?        Ss   15:09   0:00 /sbin/rpcbind
-l -w
rpc        977  0.0  0.1  17636  2572 ?        Ss   15:09   0:00
/sbin/rpc.statd
root      1018  0.0  0.0  10916   116 ?        S    15:09   0:00
/usr/sbin/rpc.yppasswdd
root      1023  0.0  0.0   8696   104 ?        S    15:09   0:00
/usr/sbin/rpc.ypxfrd
root      1060  0.0  0.0  26480   396 ?        Ss   15:09   0:00
/usr/sbin/rpc.rquotad
root      1076  0.0  0.1  21032  3904 ?        Ss   15:09   0:00
/usr/sbin/rpc.mountd
csa476t1  7512  0.0  0.1  11672  2188 pts/6    S+   18:02   0:00 grep rpc

9) What is the resident set size and total (virtual memory or VM) size of the
{\ltt{}inetd} process.

ps auxww | grep inetd

Resident Set Size: 1712
VM size: 6488

This part of the assignment is designed to familiarize you with the
commands involving virtual memory and swapping.

10) As it is set up,
does your machine use a swap partition or a swap file? 
Report which it is.
Report the full name of the swap partition or file.

free
vim /etc/fstab

/de/sda1	swap	swap	defaults	0	0

11) Report the following in kilobytes with the
command used to find it:

free -m

a) the amount of memory available.
1454 MB
\break
b) the amount of memory in use.
 213 MB
\break
c) the amount of memory used for buffers.
281 MB
\break
d) the amount of swap space available.
 95385 MB
\break
e) the amount of swap space in use.
0 MB


On the lab machine you are assigned to administer:

Set up a 4 MB (or 16 MB depending on blocksize) swap file, 
call it {\ltt{}/swapfile}. Turn on swapping for that file.

12) Report the exact commands you used to do this.

dd if=/dev/zero of=/swapfile bs=1024 count=4096   

mkswap /swapfile

swapon /swapfile


13) Report the amount of swap space available and the 
amount in use before and after turning swapping on 
(it could be zero use before and after).

Free: 15265 MB
Used: 0

Turn off swapping and remove your file.

Note for subsequent questions:
If you did the {\ltt{}fdisk} as directed in the previous 
homework, the swap partion you created should be 
{\ltt{}/dev/sdb1}; this is your ``second" swap partition.
The swap partition I created is {\ltt{}/dev/sda1} 
and is already in use.

First, set up the swap structure in your second swap partition.
Second, modify your {\ltt{}fstab} so that swapping in 
that partition is turned on automatically at boot.
(Use the {\ltt{}swapon -a} command to make sure you've got
everything setup correctly, else the next step will lock up!!!)
Now that you've tested to make sure things work, reboot
(This is to make sure the ``automatically at boot" is working.)
This new swap partition is permanent, do not remove it.

swapoff /swapfile
rm /swapfile
mkswap /dev/sdb1
vim /etc/fstab

/dev/sdb1 swap...

swapon -a

--restart--


14) Report the amount of swap space available and the amount in use.
(This should be different from your previous report of 
swap space, by approximately the size of your new 
swap partition!)

free

Free: 23442104
used: 0

\bye
